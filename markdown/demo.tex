\documentclass[ignorenonframetext,]{beamer}
\usepackage{amssymb,amsmath}
\usepackage[boldfont,slantfont,CJKsetspaces,CJKchecksingle]{xeCJK}
\newcommand{\hurl}[1]{\href{#1}{\color{blue}#1}} %自定义的命令\hurl
\usepackage{ifxetex,ifluatex}

% 表格环境
\usepackage{longtable}

% 下面三行是表格脚注的定义
\makeatletter
\def\fnum@table{\tablename~\thetable}
\makeatother


\ifxetex
\usepackage{fontspec,xltxtra,xunicode}
\defaultfontfeatures{Mapping=tex-text,Scale=MatchLowercase}
\setCJKmainfont{文泉驿微米黑}
\setCJKsansfont{Kaiti}
\setCJKmonofont{SimSun}
\punctstyle{quanjiao}
\else
\ifluatex
\usepackage{fontspec}
\defaultfontfeatures{Mapping=tex-text,Scale=MatchLowercase}
\else
\usepackage[utf8]{inputenc}
\fi
\fi




  \usepackage{color}
  \usepackage{fancyvrb}
  \DefineShortVerb[commandchars=\\\{\}]{\|}
  \DefineVerbatimEnvironment{Highlighting}{Verbatim}{commandchars=\\\{\}}
  % Add ',fontsize=\small' for more characters per line
  \newenvironment{Shaded}{}{}
  \newcommand{\KeywordTok}[1]{\textcolor[rgb]{0.00,0.44,0.13}{\textbf{{#1}}}}
  \newcommand{\DataTypeTok}[1]{\textcolor[rgb]{0.56,0.13,0.00}{{#1}}}
  \newcommand{\DecValTok}[1]{\textcolor[rgb]{0.25,0.63,0.44}{{#1}}}
  \newcommand{\BaseNTok}[1]{\textcolor[rgb]{0.25,0.63,0.44}{{#1}}}
  \newcommand{\FloatTok}[1]{\textcolor[rgb]{0.25,0.63,0.44}{{#1}}}
  \newcommand{\CharTok}[1]{\textcolor[rgb]{0.25,0.44,0.63}{{#1}}}
  \newcommand{\StringTok}[1]{\textcolor[rgb]{0.25,0.44,0.63}{{#1}}}
  \newcommand{\CommentTok}[1]{\textcolor[rgb]{0.38,0.63,0.69}{\textit{{#1}}}}
  \newcommand{\OtherTok}[1]{\textcolor[rgb]{0.00,0.44,0.13}{{#1}}}
  \newcommand{\AlertTok}[1]{\textcolor[rgb]{1.00,0.00,0.00}{\textbf{{#1}}}}
  \newcommand{\FunctionTok}[1]{\textcolor[rgb]{0.02,0.16,0.49}{{#1}}}
  \newcommand{\RegionMarkerTok}[1]{{#1}}
  \newcommand{\ErrorTok}[1]{\textcolor[rgb]{1.00,0.00,0.00}{\textbf{{#1}}}}
  \newcommand{\NormalTok}[1]{{#1}}


% Redefine labelwidth for lists; otherwise, the enumerate package will cause
% markers to extend beyond the left margin.
% \makeatletter\AtBeginDocument{%
% \renewcommand{\@listi}
% {\setlength{\labelwidth}{4em}}
% }\makeatother
  \usepackage{enumerate}

  \usepackage{ctable}
  \usepackage{float} % provides the H option for float placement

  \usepackage{url}


% Comment these out if you don't want a slide with just the
% part/section/subsection/subsubsection title:
\AtBeginPart{\frame{\partpage}}
\AtBeginSection{\frame{\sectionpage}}
\AtBeginSubsection{\frame{\subsectionpage}}
\AtBeginSubsubsection{\frame{\subsubsectionpage}}

% \setlength{\parindent}{0pt}
% \setlength{\parskip}{6pt plus 2pt minus 1pt}
% \setlength{\emergencystretch}{3em} % prevent overfull lines
  \setcounter{secnumdepth}{0}



 
      \title{R语言春令营(1):导论}
  
  \author{\href{mailto:chen@yanping.me}{陈堰平}}

  \date{April 14, 2013}

\institute[统计之都]{
统计之都\\
\hurl{http://cos.name}
}

\setbeamertemplate{navigation symbols}{}
\begin{document}
  \frame{\titlepage}
 
\section{R的安装}

\begin{frame}\frametitle{Reproducibility}

\begin{itemize}
\item
  Same script
\item
  Same results
\item
  Anywhere
  \begin{itemize}
  \item
    Single thread
  \item
    Multi-core
  \item
    Cloud Scale
  \end{itemize}
\end{itemize}
\end{frame}

\begin{frame}\frametitle{Everything starts with a seed.}

Simulation is based off Pseudo-random number generation (PRNG).

\begin{itemize}
\item
  PRNG is sequential, next number depends on the last state.
\item
  Seeds are used to store the state of a random number generator
\item
  by `Setting a seed' one can place a PRNG into any exact state.
\end{itemize}
\end{frame}

\begin{frame}\frametitle{Parallel Random Number Generation}

Simulation is complicated in new parallel environments.

\begin{itemize}
\item
  PRNG is sequential,
\item
  parallel execution is not,
\item
  and order of execution is not guaranteed.
\end{itemize}
\textbar{} Right \textbar{} Left \textbar{} Default \textbar{} Center
\textbar{}
\textbar{}------:\textbar{}:-----\textbar{}---------\textbar{}:------:\textbar{}
\textbar{} 12 \textbar{} 12 \textbar{} 12 \textbar{} 12 \textbar{}
\textbar{} 123 \textbar{} 123 \textbar{} 123 \textbar{} 123 \textbar{}
\textbar{} 1 \textbar{} 1 \textbar{} 1 \textbar{} 1 \textbar{}

: Demonstration of simple table syntax.

This is where parallel pseudo-random number generators help out.

\end{frame}

\begin{frame}\frametitle{Parallel PRNG}

Parallel pseudo-random number generators start with a singe state that
can spawn additional streams as well as streams of random numbers.

\begin{enumerate}[1.]
\item
  SPRNG
\item
  L'Ecuyer combined multiple-recursive generator
\end{enumerate}
\ctable[caption = Here's the caption., pos = H, center, botcap]{clrl}
{% notes
}
{% rows
\FL
\parbox[b]{0.17\columnwidth}{\centering
Centered Header
} & \parbox[b]{0.11\columnwidth}{\raggedright
Default Aligned
} & \parbox[b]{0.22\columnwidth}{\raggedleft
Right Aligned
} & \parbox[b]{0.35\columnwidth}{\raggedright
Left Aligned
}
\ML
\parbox[t]{0.17\columnwidth}{\centering
First
} & \parbox[t]{0.11\columnwidth}{\raggedright
row
} & \parbox[t]{0.22\columnwidth}{\raggedleft
12.0
} & \parbox[t]{0.35\columnwidth}{\raggedright
Example of a row that spans multiple lines.
}
\\\noalign{\medskip}
\parbox[t]{0.17\columnwidth}{\centering
Second
} & \parbox[t]{0.11\columnwidth}{\raggedright
row
} & \parbox[t]{0.22\columnwidth}{\raggedleft
5.0
} & \parbox[t]{0.35\columnwidth}{\raggedright
Here's another one. Note the blank line between rows.
}
\LL
}

\end{frame}

\section{R包的使用}

\begin{frame}[fragile]\frametitle{R package \texttt{harvestr}}

\url{https://github.com/halpo/harvestr}

What \texttt{harvestr} does:

\begin{itemize}
\item
  Reproducibility
\item
  Caching
\item
  Under parallelized environments.
\end{itemize}
\end{frame}

\begin{frame}\frametitle{How \texttt{harvestr} works}

\begin{itemize}
\item
  Analytical elements are separated into work-flows of dependent
  elements.
  \begin{itemize}
  \item
    Set up environment/seed
  \item
    Generate Data
  \item
    Perform analysis
    \begin{itemize}
    \item
      Stochastic
    \item
      Non-Stochastic
    \end{itemize}
  \item
    Summarize
  \end{itemize}
\item
  Results from one step carry to another by carrying the seed with the
  results.
\end{itemize}
\end{frame}

\begin{frame}[fragile]\frametitle{\textbf{Primary work-flow} for
\texttt{harvestr}}

\begin{itemize}
\item
  \texttt{gather(n)} - generate \texttt{n} random number streams.
\item
  \texttt{farm(seeds, expr)} - evaluate \texttt{expr} with each seed in
  \texttt{seeds}.
\item
  \texttt{harvest(x, fun)} - for each data in \texttt{x} call the
  function \texttt{fun}
\end{itemize}
\end{frame}

\section{数据管理}

\begin{frame}[fragile]\frametitle{Building blocks}

Some building blocks that might \emph{might} be helpful.

\begin{itemize}
\item
  \texttt{plant}- for setting up copies of an object with given seeds.
\item
  \texttt{sprout} - for obtaining the sub-streams used with graft.
\item
  \texttt{reap} - single object version of \texttt{harvest}
\end{itemize}
\end{frame}

\begin{frame}[fragile]\frametitle{In case you are wondering}

\begin{itemize}
\item
  Yes it works with \texttt{Rcpp} code,
\item
  provided the compiled code uses the RNGScope for RNG in C++.
\item
  \textbf{But} take care to not carry C++ reference objects across
  parallel calls.
\end{itemize}
\begin{Shaded}
\begin{Highlighting}[]
\NormalTok{seeds <- }\KeywordTok{gather}\NormalTok{(}\DecValTok{10}\NormalTok{, }\DataTypeTok{seed=}\DecValTok{20120614}\NormalTok{)}
\NormalTok{data <- }\KeywordTok{farm}\NormalTok{(seeds, \{}
\NormalTok{x1 <- }\KeywordTok{rnorm}\NormalTok{(}\DecValTok{400}\NormalTok{)}
\NormalTok{x2 <- }\KeywordTok{rnorm}\NormalTok{(}\DecValTok{400}\NormalTok{)}
\NormalTok{g <- }\KeywordTok{rep}\NormalTok{(}\KeywordTok{rnorm}\NormalTok{(}\DecValTok{4}\NormalTok{), }\DataTypeTok{each=}\DecValTok{100}\NormalTok{)}
\NormalTok{trt <- }\KeywordTok{rep}\NormalTok{(}\DecValTok{1}\NormalTok{:}\DecValTok{4}\NormalTok{, }\DataTypeTok{each=}\DecValTok{100}\NormalTok{)}
\NormalTok{y <- }\KeywordTok{rnorm}\NormalTok{(}\DataTypeTok{n=}\DecValTok{400}\NormalTok{, }\DataTypeTok{mean=}\DecValTok{3}\NormalTok{*x1 + x2 + g)}
\KeywordTok{data.frame}\NormalTok{(y, x1, x2, trt)}
\NormalTok{\})}
\end{Highlighting}
\end{Shaded}
\end{frame}
 

\end{document}